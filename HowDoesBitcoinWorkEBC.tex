
\chapter{How Does Bitcoin Work?}

The essential principle is that every user has a piece of software
on their computer, a\textbf{ \nomenclature{wallet}{ }}, which contains
a set of public addresses (like bank account numbers), each address
is mathematically linked with a \textbf{\nomenclature{private key}{ }}
(like a password). Using this secret key, users can create a digital
\nomenclature{signature}{ }, to prove that they are the owner of
an address. This signature, together with a transaction is sent to
a bitcoin \textbf{\nomenclature{miner}{ }}. The miner is a node in
the network, which verifies that the address and signature are linked,
after which the transaction is recorded in the public ledger, or \textbf{\nomenclature{blockchain}{ }}
and disseminated throughout the network.

There are a number of other features which are important to highlight.
Bitcoins are created through a process called mining, this is a computationally
intensive process used as the mechanism for transaction verification.
The Bitcoins created as a reward for mining becomes incrementally
smaller, until finally becoming zero. This is predicted to be around
the year 2140, and at that point around 21 million bitcoins will have
been created. There will never be more bitcoins in the system. Furthermore,
bitcoins will be lost if private keys are lost, these will never be
recovered. The system is thus strictly deflationary. To keep transactions
of every size possible, bitcoins are highly granular, every bitcoin
can be divided into a hundred million \textbf{\nomenclature{satoshi}{ }}
(named after the pseudonym of the creator).
