
\chapter{Cryptocurrencies and Inflation}

As the value of a \nomenclature{bitcoin}{ } goes up, more people
will choose to engage in the lucrative \nomenclature{mining}{ }.
As extra computational power comes in, the difficulty of \nomenclature{mining}{ }
automatically goes up, which means more computing power is used for
the same transactions. This is necessary to safeguard the integrity
of the network.

Having established the key features of the \nomenclature{Bitcoin}{ }
system, we will now focus on two often heard economic concerns about
the Bitcoin system. Namely the issue of \nomenclature{deflation}{ }
and the issue of socially wasteful \nomenclature{mining}{ }.

As mentioned above, there will only ever be around 21 million bitcoins,
and some will be lost, this makes the system deflationary, which is
troubling to economists. A \nomenclature{deflation}{ } in a currencies
value provides a disincentive to spend, causing economic slowdown
\citep[see e.g.][]{fisher1933debt}. Since deflation causes the price
of products to fall, it incentivises people to save ans spend later.
Additionally, these savings are not invested properly, since deflation
simultaneously provides a disincentive for borrowers, by making future
paybacks more expensive, raising the effective interest rate.

Inflation is impossible within the Bitcoin network, as well as within
most other cryptocurrencies. However, inflation in the cryptocurrency
economy is still possible, through an expansion the number of cryptocurrencies.
