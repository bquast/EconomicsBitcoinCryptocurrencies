
\chapter{Introduction}

In 2009 the online currency\textbf{ \nomenclature{Bitcoin}{ }} was
launched by an anonymous developer known by the pseudonym \nomenclature{Satoshi Nakamoto}{ }.
Initially the project remained fairly low key, have only a small group
of geeks participate in it. However, in 2013 Bitcoin quickly became
a mainstream phenomenon as it was reported on in the mainstream media
such as The Economics, Time Magaine, The New York Times, and many
more. This increased attention created a surge in the uptake of Bitcoin
both as an investment as well as transaction mechanism. However, Bitcoin
also has a dark side, being initially used mostly to purchase illicit
goods on online marketplaces such The Silk Road, websites that are
generally only available through the dark web. However, the rise in
popular attention has also brought the attention of regulators, which
has caused many of the illegal marketplaces to be taken offline, as
well as introduced a range of fiscal regulation, dealing with bitcoin.
This book provides a laymens introduction to the basic principles
of Bitcoin and other online currencies, as well as discuss how this
new phenomenon can be viewed from an economic perspective, and what
that would mean for the development of such systems.

The Bitcoin system was based on a 2008 white paper by the same author
\citep{nakamoto2008bitcoin}. 

Besides the fact that Bitcoin exist only digitally, its defining characteristics
are decentralisation and openness. and every aspect of it is public
and \nomenclature{open source}{ }. 

This quickly led to the launching of several alternative online currencies,
based on the Bitcoin protocol. A currency based on cryptography, such
as Bitcoin, is know as a \nomenclature{cryptocurrency}{ }.

Cryptocurrencies come with an extensive new terminology, for reference,
a glossary is provided in the back of the book.
