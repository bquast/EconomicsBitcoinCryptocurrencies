
\chapter{What is Bitcoin?}

A short answer is that it is an online anonymous currency which is
not controlled by a government or any single entity. However, from
a technical as well as a from a social perspective there are many
innovative features to \nomenclature{Bitcoin}{  }. For instance,
users hold his funds himself. Meaning that there is no need for a
middleman, such as a bank. This prevents fees and eliminates risks
associated with bank insolvency. 

\nomenclature{Bitcoin}{  }is built from the ground up to be decentralized,
anonymous, and openly verifiable. All software is open source and
publicly available. Anybody is free to perform any function within
the network (user or node), there is no central or essential node,
there is redundancy in every aspect.

\nomenclature{Bitcoin}{ }is a confusing term. It is both a monetary
unit of denomination, such as Swiss franc, euro, or pound sterling,
as well as, the monetary system as a whole, including the protocol,
the network, and all participants. To distinguish between the two,
the monetary system \nomenclature{Bitcoin}{  } is written with a
capital letter B, whereas the monetary unit of account is \nomenclature{bitcoin}{ },
is written with a small letter b. Besides these two, there is the
original \nomenclature{Bitcoin}{  } software, which is called\textbf{
\nomenclature{Bitcoin-Qt}{ }}, using this software package, or others,
computer can connect to the \nomenclature{Bitcoin}{  } network and
perform transactions.
